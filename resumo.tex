O trânsito é um elemento presente na rotina de todas as pessoas que circulam por ambientes urbanos, seja como motorista, passageiro ou pedestre. Para assegurar segurança e boa fluidez na locomoção dessas pessoas, é necessário que haja um controle dos aspectos que ditam os fluxos nas vias.
Esse trabalho apresenta um conjunto de sistemas, com funcionalidades específicas, que integrados em um único equipamento, têm por funcionalidade o controle de um cruzamento veicular, funcionando de forma autônoma, após configurados. O equipamento é capaz de realizar o controle em modo de tempos fixos, pré-definidos, e em modo adaptativo, no qual o sistema se adapta ao trânsito, em tempo real.

Além de apresentar o equipamento responsável pelo controle semafórico, foram realizadas simulações para demonstrar o impacto causado em um cruzamento com fluxos veiculares em diferentes sentidos. As simulações foram realizadas com quatro cenários distintos, representando diferentes modos de funcionamento do equipamento.
Os resultados obtidos das simulações demonstraram um impacto positivo do equipamento no cruzamento semafórico, tanto em funcionamento de tempos fixos, como em modo adaptativo de tempo real.

