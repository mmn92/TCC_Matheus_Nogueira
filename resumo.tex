A iluminação de LED apresenta algumas vantagens diante das tradicionais iluminações fluorescentes e incandescentes. Lâmpadas ou fitas de LED são em geral mais eficientes, compactas e oferecem vantagens como variedade de cores em comparação às variantes tradicionais de iluminação e por isso vêm sendo utilizadas tanto em iluminação pública como em decoração e em pequenos dispositivos. A facilidade de controlar sua intensidade por meio do rápido chaveamento de sua potência é outro atrativo, principalmente quando se trata do uso de microcontroladores conectados à internet. Este trabalho apresenta uma implementação de um sistema de iluminação com fita de LEDs brancos controlada pelo sistema embarcado "ESP-8266" por meio da troca de dados pela internet com um aplicativo móvel por meio do protocolo de internet "MQTT" com a capacidade de ligar, desligar, arbitrar a intensidade da fita no modo manual ou ter a intensidade controlada automaticamente. Serão feitas medições do consumo do sistema para demonstrar a capacidade de economia de energia do sistema proposto.
