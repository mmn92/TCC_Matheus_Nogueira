Traffic is an element in the routine of all people that circulate in urban environments, whether as a driver, passenger or pedestrian. In order to ensure safety and good fluidity in the locomotion of these people, it is necessary that there is a control of the aspects that dictate the flows in the streets.
This work presents a set of systems, with specific functionalities, that are integrated into a single equipment, with the purpose of controlling a vehicular crossing, working autonomously, after configured. The equipment is able to perform the control in a preset fixed-time mode, or in adaptive mode, in which the system adapts to the traffic load, in real time.

In addition to presenting the equipment responsible for traffic lights control, simulations were carried out to demonstrate the impact caused by installing the system on an area with vehicle traffic flowing in different directions. The simulations were performed with four different scenarios, representing different modes of operation of the equipment.
The results obtained from the simulations demonstrated a positive impact of the equipment on the traffic, both in fixed time operation and in real time adaptive mode.